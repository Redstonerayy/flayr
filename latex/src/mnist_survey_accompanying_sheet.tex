\documentclass[a4paper]{article}

% change to german
\usepackage[german]{babel}

% better hyphenation
\usepackage[final]{microtype}
\usepackage{csquotes}

% packages, images, math
\usepackage{geometry, graphicx, amsfonts, multicol}

% font
\usepackage[scaled]{helvet}
\usepackage[T1]{fontenc}

% fontsize
\usepackage[14pt]{extsizes}

% Set Layout
\geometry{
    a4paper,
    left=25mm,
    right=25mm,
    top=15mm,
    bottom=25mm
}

% import line spacing
\usepackage{setspace}

% set line spacing
\setstretch{1.5}

\begin{document}

\author{Anton Rodenwad}
\title{Erstellung eines eigenen Datensets zur Handschriftenerkennung mit KI}
\date{}

\maketitle

\thispagestyle{empty}


\section{Motivation}
Um eine künstliche Intelligenz zu entwicklen, die Handschriften erkennen kann,
braucht man Trainingsdaten. Nur wenn man solche Daten hat, kann man
anschließend die KI trainieren. Deswegen sammeln wir Daten,
um ein eigenes Datenset aufzubauen.
Wer interesse an diesem Datenset hat, kann sich gerne melden.
Es wird frei auf Github (Filesharing Seite) verfügbar sein.

\section{Erstellung des Datensatzes}
Danke erstmal, dass ihr uns helft, Daten für unser Datenset zu sammeln.
Wenn ihr jetzt gleich die Zettel ausfüllt, achtet bitte darauf, möglichst
innerhalb der Grenzen der Kästchen zu bleiben.
Auf dem folgenden Beispielblatt ist gezeigt, wie ein ausgefülltes Blatt
aussehen sollte (stellt euch einfach vor, es wäre handschriftlich ausgefüllt
worden).

\end{document}
